%\VignetteIndexEntry{DemoPackage Overview}
%\VignetteDepends{DemoPackage}
%\VignettePackage{DemoPackage}

\documentclass[12pt]{article}

\usepackage{times}
\usepackage{hyperref}

\usepackage[authoryear,round]{natbib}
\usepackage{times}
\usepackage{comment}

\textwidth=6.2in
\textheight=8.5in
%\parskip=.3cm
\oddsidemargin=.1in
\evensidemargin=.1in
\headheight=-.3in

\newcommand{\scscst}{\scriptscriptstyle}
\newcommand{\scst}{\scriptstyle}

\newcommand{\Rfunction}[1]{{\texttt{#1}}}
\newcommand{\Robject}[1]{{\texttt{#1}}}
\newcommand{\Rpackage}[1]{{\textit{#1}}}
\newcommand{\Rclass}[1]{{\textit{#1}}}


\bibliographystyle{plainnat}


\title{DemoPackage}
\author{Paul Shannon}

\usepackage{/Library/Frameworks/R.framework/Resources/share/texmf/tex/latex/Sweave}
\begin{document}

\maketitle

\emph{DemoPackage} is a nearly empty R package for you, the R programmer, to adopt and adapt to your own purposes.  

\section{More Help}
Vignettes are an amalgamation of R code and TeX documents.  They can be bewildering.  A possibly useful example is the one I wrote for the RCytoscape package, which can be obtained using these steps:

\begin{itemize}
  \item Browse to \url{http://bioconductor.org/packages/release/bioc/html/RCytoscape.html}
  \item Download \emph{Package Source}
  \item Extract the vignette file (but note that the version number will very likely be different):
\begin{verbatim}
tar zxf RCytoscape_1.2.1.tar.gz RCytoscape/inst/doc/RCytoscape.Rnw
\end{verbatim}
  \item Study and/or steal from this file: \emph{RCytoscape.Rnw}
\end{itemize}



\end{document}
